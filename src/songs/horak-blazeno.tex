\begin{song}{
    title=Blaženo,
    interpret=Michal Horák
}

\begin{verse}
^{G}Tenhle příběh fakt se stal u nás na vesnici, \\
když místní báby vítaly novou kolchoznici. \\
Pojď Blaženo tajně vezmem tě v lese do míst, \\
kde páry parkujou své káry, když chtěj soukromí.
\end{verse}

\begin{chorus}
^{G}Hej ty ^{C}Blaženo, \\
^{G}já špatnej pocit ^{D}mám. \\
^{G \z H7}Okýnko je ^{Em \z C}staženo \\  
^{G}však radši ^{D}nechoď ^{G}tam.
\end{chorus}

\begin{verse}
Koukej Blážo, máme štěstí, auto už je tu, \\
tipuju to na Karla a jeho Markétu. \\
Blažena se zděsila: \uv{No tohle je snad šprým? \\
Co je to v káře za hambáře, já jim domluvím.}
\end{verse}

\begin{chorus}
\end{chorus}

\begin{chorus}
\end{chorus}

\begin{verse}
Blažena se plíživě u auta skrčila, \\
pak otevřeným okýnkem tam hlavu strčila. \\
Pak je oba seřvala až otřásal se džíp, \\
Karel rychle okno zavřel, Bláže hlavu skříp.
\end{verse}

\begin{verse*}
\uv{Když kde nemáš co dělat stále jen se potloukáš, \\
tak, ty bábo zvědavá, a teď to dokoukáš.}
\end{verse*}

\begin{interlude}
/: _{Em} _{Em} _{D} _{A} :/
\end{interlude}

\begin{verse}
Asi po půl hodině se Bláža vrací zpět, \\
kolchoznice chtějí hnedka všechno povědět. \\
\uv{To byste mi nevěřily, co já viděla.} \\
A do lesa na palouk už nikdy nechtěla.
\end{verse}

\begin{chorus}
\end{chorus}

\begin{chorus}
^{G}Hej ty ^{C}Blaženo, \\
^{G}já špatnej pocit ^{D}mám. \\
... ^{G}až příště ^{H7}okno bude ^{Em \z C}staženo, \\
^{G}tak nikdo ^{D}nechoď ^{G}tam!
\end{chorus}

\end{song}